
\documentclass[journal,12pt,twocolumn]{IEEEtran}

\usepackage{setspace}
\usepackage{gensymb}

\singlespacing


\usepackage[cmex10]{amsmath}

\usepackage{amsthm}

\usepackage{mathrsfs}
\usepackage{txfonts}
\usepackage{stfloats}
\usepackage{bm}
\usepackage{cite}
\usepackage{cases}
\usepackage{subfig}

\usepackage{longtable}
\usepackage{multirow}

\usepackage{enumitem}
\usepackage{mathtools}
\usepackage{steinmetz}
\usepackage{tikz}
\usepackage{circuitikz}
\usepackage{verbatim}
\usepackage{tfrupee}
\usepackage[breaklinks=true]{hyperref}
\usepackage{graphicx}
\usepackage{tkz-euclide}

\usetikzlibrary{calc,math}
\usepackage{listings}
    \usepackage{color}                                            %%
    \usepackage{array}                                            %%
    \usepackage{longtable}                                        %%
    \usepackage{calc}                                             %%
    \usepackage{multirow}                                         %%
    \usepackage{hhline}                                           %%
    \usepackage{ifthen}                                           %%
    \usepackage{lscape}     
\usepackage{multicol}
\usepackage{chngcntr}

\DeclareMathOperator*{\Res}{Res}

\renewcommand\thesection{\arabic{section}}
\renewcommand\thesubsection{\thesection.\arabic{subsection}}
\renewcommand\thesubsubsection{\thesubsection.\arabic{subsubsection}}

\renewcommand\thesectiondis{\arabic{section}}
\renewcommand\thesubsectiondis{\thesectiondis.\arabic{subsection}}
\renewcommand\thesubsubsectiondis{\thesubsectiondis.\arabic{subsubsection}}


\hyphenation{op-tical net-works semi-conduc-tor}
\def\inputGnumericTable{}                                 %%

\lstset{
%language=C,
frame=single, 
breaklines=true,
columns=fullflexible
}
\begin{document}


\newtheorem{theorem}{Theorem}[section]
\newtheorem{problem}{Problem}
\newtheorem{proposition}{Proposition}[section]
\newtheorem{lemma}{Lemma}[section]
\newtheorem{corollary}[theorem]{Corollary}
\newtheorem{example}{Example}[section]
\newtheorem{definition}[problem]{Definition}

\newcommand{\BEQA}{\begin{eqnarray}}
\newcommand{\EEQA}{\end{eqnarray}}
\newcommand{\define}{\stackrel{\triangle}{=}}
\bibliographystyle{IEEEtran}
\providecommand{\mbf}{\mathbf}
\providecommand{\pr}[1]{\ensuremath{\Pr\left(#1\right)}}
\providecommand{\qfunc}[1]{\ensuremath{Q\left(#1\right)}}
\providecommand{\sbrak}[1]{\ensuremath{{}\left[#1\right]}}
\providecommand{\lsbrak}[1]{\ensuremath{{}\left[#1\right.}}
\providecommand{\rsbrak}[1]{\ensuremath{{}\left.#1\right]}}
\providecommand{\brak}[1]{\ensuremath{\left(#1\right)}}
\providecommand{\lbrak}[1]{\ensuremath{\left(#1\right.}}
\providecommand{\rbrak}[1]{\ensuremath{\left.#1\right)}}
\providecommand{\cbrak}[1]{\ensuremath{\left\{#1\right\}}}
\providecommand{\lcbrak}[1]{\ensuremath{\left\{#1\right.}}
\providecommand{\rcbrak}[1]{\ensuremath{\left.#1\right\}}}
\theoremstyle{remark}
\newtheorem{rem}{Remark}
\newcommand{\sgn}{\mathop{\mathrm{sgn}}}
\providecommand{\abs}[1]{\left\vert#1\right\vert}
\providecommand{\res}[1]{\Res\displaylimits_{#1}} 
\providecommand{\norm}[1]{\left\lVert#1\right\rVert}
%\providecommand{\norm}[1]{\lVert#1\rVert}
\providecommand{\mtx}[1]{\mathbf{#1}}
\providecommand{\mean}[1]{E\left[ #1 \right]}
\providecommand{\fourier}{\overset{\mathcal{F}}{ \rightleftharpoons}}
%\providecommand{\hilbert}{\overset{\mathcal{H}}{ \rightleftharpoons}}
\providecommand{\system}{\overset{\mathcal{H}}{ \longleftrightarrow}}
	%\newcommand{\solution}[2]{\textbf{Solution:}{#1}}
\newcommand{\solution}{\noindent \textbf{Solution: }}
\newcommand{\cosec}{\,\text{cosec}\,}
\providecommand{\dec}[2]{\ensuremath{\overset{#1}{\underset{#2}{\gtrless}}}}
\newcommand{\myvec}[1]{\ensuremath{\begin{pmatrix}#1\end{pmatrix}}}
\newcommand{\mydet}[1]{\ensuremath{\begin{vmatrix}#1\end{vmatrix}}}
\numberwithin{equation}{subsection}
\makeatletter
\@addtoreset{figure}{problem}
\makeatother
\let\StandardTheFigure\thefigure
\let\vec\mathbf
\renewcommand{\thefigure}{\theproblem}
\def\putbox#1#2#3{\makebox[0in][l]{\makebox[#1][l]{}\raisebox{\baselineskip}[0in][0in]{\raisebox{#2}[0in][0in]{#3}}}}
     \def\rightbox#1{\makebox[0in][r]{#1}}
     \def\centbox#1{\makebox[0in]{#1}}
     \def\topbox#1{\raisebox{-\baselineskip}[0in][0in]{#1}}
     \def\midbox#1{\raisebox{-0.5\baselineskip}[0in][0in]{#1}}
\vspace{3cm}
\title{Assignment-5}
\author{Ankur Aditya - EE20RESCH11010}
\maketitle
\newpage
\bigskip
\renewcommand{\thefigure}{\theenumi}
\renewcommand{\thetable}{\theenumi}

\begin{abstract}
This document contains the procedure to find unknown constant value such that the equation represents pair of straight lines.
\end{abstract}
Download the python code from 
\begin{lstlisting}
https://github.com/ankuraditya13/EE5609-Assignment5
\end{lstlisting}
%
and latex-file codes from 
%
\begin{lstlisting}
https://github.com/ankuraditya13/EE5609-Assignment5
\end{lstlisting}

\section{Problem}
Find the value of k so that following equation may represent pairs of straight lines,
\begin{align}
kxy-8x+9y-12 = 0
\label{Q_eq}
\end{align}
\section{Solution}
The general equation of second degree is given by,
\begin{align}
ax^2 + 2bxy + cy^2 + 2dx +2ey +f = 0
\label{main_eq}
\end{align}
In vector from the equation \eqref{main_eq} canb be expressed as,
\begin{align}
\vec{x}^T\vec{V}\vec{x} + 2\vec{u}^T\vec{x} + f = 0 
\label{line_vec}
\end{align}
where,
\begin{align}
\vec{V} = \vec{V}^T = \myvec{a&b\\b&c}
\label{vec1}
\end{align}
\begin{align}
\vec{u} = \myvec{d\\e}
\label{vec2}
\end{align}
Now, comparing equation \eqref{main_eq} to \eqref{Q_eq} we get, a = c = 0, b = $\brak{\frac{k}{2}}$, d = -4, e = $\brak{\frac{9}{2}}$, f = -12.  
Hence, substituting these values in equation \eqref{vec1} and \eqref{vec2} we get,
\begin{align}
\vec{V} = \vec{V}^T = \myvec{0&\frac{k}{2}\\\frac{k}{2}&0}
\end{align}
\begin{align}
\vec{u} = \myvec{-4\\\frac{9}{2}}
\end{align}
Now equation \eqref{Q_eq} represents pair of straight lines if,
\begin{align}
\mydet{\vec{V}&\vec{u}\\\vec{u}^T&f} = 0
\end{align}
\begin{align}
\mydet{0&\frac{k}{2}&-4\\\frac{k}{2}&0&\frac{9}{2}\\-4&\frac{9}{2}&-12} = 0
\end{align}
\begin{align}
\implies k =0 , k = 6 
\label{k}
\end{align}
Substitutong \eqref{k} in \eqref{Q_eq} we get,
\begin{align}
6xy-8x+9y-12 = 0
\end{align}
\begin{align}
-8x+9y-12 = 0
\end{align}
Hence value of k = 6 represents pair of straight lines.
Also it can be verified that the pair of lines intersect as,
\begin{align}
\mydet{\vec{V}} = \mydet{0&3\\3&0} <0 
\end{align} 
Let the pair of straight lines is given by,
\begin{align}
\vec{n_1}^T\vec{x} = c1
\label{e1}
\end{align}
\begin{align}
\vec{n_2}^T\vec{x} = c2
\label{e2}
\end{align}
Now equating the product of equation \eqref{e1} and \eqref{e2} with \eqref{line_vec} we get,
\begin{align}
(\vec{n_1}^T\vec{x} - c1)(\vec{n_2}^T\vec{x} - c2) = \vec{x}^T\myvec{0&3\\3&0}\vec{x} + 2\myvec{-4& \frac{9}{2}}\vec{x} - 12 
\end{align}
\begin{align}
\implies n_1 * n_2 = \{0,6,0\}
\end{align}
\begin{align}
c_1n_1+c_2n_2 = \myvec{8\\-9}
\end{align}
\begin{align}
c_1c_2 = -12.
\end{align}
Now the slopes of line is given by roots of polynomial,
\begin{align}
cm^2 + 2bm + a = 0
\end{align} 
\begin{align}
\implies 2bm = 0
\end{align}
\begin{align}
\implies m = 0
\end{align}

\end{document}
